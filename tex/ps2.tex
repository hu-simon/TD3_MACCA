\NeedsTeXFormat{LaTeX2e}[1994/06/01]
\ProvidesPackage{ps}[09/26/2019 v1.0 LaTeX style package for the purposes of writing homework sets]

\RequirePackage{}
\let\myDate\date

\usepackage[utf8]{inputenc}

% Packages for styling options.
% Packages for figures. 
\usepackage{graphicx}
\usepackage{caption}
\usepackage{subcaption}

% Packages from AMS.
\usepackage{amsmath}
\usepackage{amssymb}
\usepackage{amsthm}

% Misc. packages.
\usepackage{physics}
\usepackage{commath}
\usepackage{mathtools}
\usepackage[colorlinks=trye, bookmarksdepth=0, hidelinks]{hyperref}

% amsthm package setup.
\theoremstyle{plain}
\newtheorem{theorem}{Theorem}
\newtheorem{prop}{Proposition}
\newtheorem{corr}{Corollary}
\newtheorem{lemma}{Lemma}

\theoremstyle{definition}
%\newtheorem{def}{Definition}
\newtheorem{example}{Example}

\theoremstyle{remark}
\newtheorem{remark}{Remark}[section]
\newtheorem*{remark*}{Remark}

\numberwithin{theorem}{section}
\numberwithin{prop}{section}
\numberwithin{remark}{section}
\numberwithin{corr}{section}
\numberwithin{lemma}{section}
%\numberwithin{def}{section}
\numberwithin{equation}{section}

% Misc. operations.
\usepackage[numbers,sort&compress]{natbib}
\bibpunct{[}{]}{;}{n}{,}{,}
\makeatletter
\def\@tocline#1#2#3#4#5#6#7{\relax
        \ifnum #1>\c@tocdepth % then omit
        \else
        \par \addpenalty\@secpenalty\addvspace{#2}%
        \begingroup \hyphenpenalty\@M
        \@ifempty{#4}{%
                \@tempdima\csname r@tocindent\number#1\endcsname\relax
        }{%
                \@tempdima#4\relax
        }%
        \parindent\z@ \leftskip#3\relax \advance\leftskip\@tempdima\relax
        \rightskip\@pnumwidth plus4em \parfillskip-\@pnumwidth
        #5\leavevmode\hskip-\@tempdima
        \ifcase #1
        \or\or \hskip 1em \or \hskip 2em \else \hskip 3em \fi%
        #6\nobreak\relax
        \hfill\hbox to\@pnumwidth{\@tocpagenum{#7}}\par% <---- \dotfill -> \hfill
        \nobreak
        \endgroup
        \fi}
\makeatother

\setcounter{tocdepth}{5}

\allowdisplaybreaks

% Macro definitions. 
\newcommand{\bfi}{\bfseries\itshape}
\newcommand{\hookuparrow}{\mathrel{\rotatebox[origin=c]{90}{$\hookrightarrow$}}}

\def\Xint#1{\mathchoice
        {\XXint\displaystyle\textstyle{#1}}%
        {\XXint\textstyle\scriptstyle{#1}}%
        {\XXint\scriptstyle\scriptscriptstyle{#1}}%
        {\XXint\scriptscriptstyle\scriptscriptstyle{#1}}%
        \!\int}
\def\XXint#1#2#3{{\setbox0=\hbox{$#1{#2#3}{\int}$}
                \vcenter{\hbox{$#2#3$}}\kern-.5\wd0}}
\def\ddashint{\Xint=}
\def\dashint{\Xint-}

\DeclarePairedDelimiterX{\inp}[2]{\langle}{\rangle}{#1, #2}

\newcommand{\R}{\mathbb{R}}
\newcommand{\F}{\mathbb{F}}
\newcommand{\C}{\mathbb{C}}
\newcommand{\V}{\mathcal{V}}
\newcommand{\bx}{\mathbf{x}}
